\documentclass{project}
\usepackage[pdfauthor={L Jones},pdftitle={Software Engineering Group Project, Interaction and high level design for the system},pdftex]{hyperref}
\usepackage{graphicx}
\usepackage[export]{adjustbox}
\usepackage{longtable}
\graphicspath{ {images} }
\begin{document}
\title{Software Engineering Group Project}
\subtitle{Test Specification}
\author{L. Jones, T. Oram, W. Jones, T. MIlls}     
\shorttitle{Test Specification}
\version{1.1}
\status{Release}
\date{2016-02-14}
\configref{SE-12-TS}
\maketitle
\tableofcontents
\newpage
\section{INTRODUCTION}
\subsection{Purpose of this document}
The purpose of this document is to outline how we aim to enforce good testing practice, and as such contains a 'Test Specification' which demonstrates how to test against the different requirements as specified in the requirements specification\cite{se.qa.rs}, as well as displaying the criteria that must be met in order to pass tests.
\subsection{Scope}
This document specifically covers the 'Test Specification' section of the Test Procedure Standards Quality Assurance Document\cite{se.qa.tps}\\
\newline
This document should be read by all project members. It is assumed that the reader is already familiar with the Test Procedure Standards \cite{se.qa.tps}.
\subsection{Objectives}
The objectives of this document are as follows:
\begin{itemize}
	\item Identify where a test will be necessary against the requirements specification{se.qa.rs} and give it a unique ID. 
	\item Explain the exact function/requirement that is being tested.
	\item Explain how a user would undertake a test, as well as the desired output if successful.
	\item Explain the criteria that must be met by the test in order for it to be deemed as "Passed".
\end{itemize} 
\clearpage
\section{TEST SPECIFICATION}
\begin{longtable}{| p{1.8cm} | p{1.1cm} | p{2.8cm} | p{3.2cm} | p{3.2cm} | p{2.8cm} |}
\hline
Test Reference & Reference Requirement & Test Content & Input & Output & Pass Criteria \\ 
\hline
MAN-001 & FR3 & TaskerMAN should support the creation of team member data & In the Create user page of taskerMAN create a new user with name ``John Smith" and email ``js@aber.ac.uk"
 & The Show users page in TaskerMAN will now include the user ``John Smith" &All data entered should be correct and the show team member's page displays js@aber.ac.uk \\
\hline
MAN-002 & FR3 & TaskerMAN should support the creation of team member data & In the create team member page of TaskerMAN try to enter in ``hello" as the email address.  & Client side form validation should execute. & Form validation should notify user that it is not a valid email address. \\
\hline
MAN-003 & FR3 & TaskerMAN should support the creation of  team member data & In the Create team member page of TaskerMAN, create a team member with their first name as ``DROP TABLE User"  & An SQL query will be executed with the first name set as ``DROP TABLE User?. & Remote database will still be intact if protection against SQL injection is correctly set up. \\
\hline
MAN-004 & FR3 & TaskerMAN should support the updating of team member data & Under the edit team member page of TaskerMAN select to edit ``js@smith.com" and change its name to ``Tom Smith".   & The show team members page in TaskerMAN should now have the updated team member  & Form validation should notify user that it is not a valid email address.\\
\hline
MAN-005 & FR3 & TaskerMAN should support the updating of team member data.  & Under the edit team member page of TaskerMAN edit ``js@smith.com"'s email to be ``hello".  
 & Client side form validation should execute.  & The show tasks page in TaskerMAN will show the updated task with it's new title\\
\hline
MAN-006 & FR3 & TaskerMAN should support the deletion of team member data.  & Under the edit team member page of TaskerMAN delete the team member with the name ``Tom Smith" & The team member will be deleted from the database. & The show team member's page in TaskerMAN should no longer include the team member ``js@smith.com" \\
\hline
MAN-007 & FR3 & TaskerMAN should support the creation of task data. & In the create team Member page of TaskerMAN create a new team member with name ``John Smith" and email ``js@aber.ac.uk" (Test subject needed) & Query will be executed to Insert into the team member table the new team member. & Database should now include the team member ``js@aber.ac.uk."\\
\hline
MAN-008 & FR4 & TaskerMAN should support the updating of task data & Under the create task page of TaskerMAN create a task with: Title = ``make coffee", Start Date =``30/01/2016",
End Date = ``1/02/2016", TaskElement1 title = ``get Cup", TaskElement2 title= ``grind beans", and allocate it to ``js@smith.com" & The task should be added to the remote database.  & The show tasks page in TaskerMAN will show the task entered and should have the exact input data and have a Task Status of ?ALLOCATED?  and will be allocated to ``js@smith.com?.  \\
\hline
MAN-009 & FR4 & TaskerMAN should support the creation of task data &Under the create task page of TaskerMAN create a task with: Title = ``make coffee", Start Date =``30/01/2016",
End Date = ``1/01/2000", TaskElement1 title = ``get Cup", TaskElement2 title= ``grind beans", and allocate it to ``js@smith.com" & Client Side form validation should execute. & User should be notified that they entered an end date that was before the start date.  \\
\hline
MAN-010 & FR4 & TaskerMAN should support the creation of task data (Testing against SQL Injection) & Under the create task page of TaskerMAN create a task with: Title = ``DROP TABLE Tasks", Start Date =``30/01/2016", End Date = ``1/02/2016", TaskElement1 title = ``get Cup", TaskElement2 title= ``grind beans", and allocate it to ``js@smith.com" & SQL Query that partly contains the String ``DROP TABLE Tasks" will be executed & Database should still be intact. \\
\hline
MAN-011 & FR4 & TaskerMAN should support the updating of task data. & Under the edit task page in TaskerMAN select the ``Make Coffee" task. Choose to add an extra task element: Title= ``pour water". & The task should be updated in the database & The show tasks page in TaskerMAN will show the updated task with an extra Task Element. \\
\hline
MAN-012 & FR4 & TaskerMAN should support the deletion of task data. & Under the edit task page in TaskerMAN, select the task ``make coffee". Delete this task.  & The selected task will be deleted from the database. & The show tasks page in TaskerMAN should now not include the task ``make coffee" as it should've been removed from the remote database. \\
\hline
MAN-013 & FR5 & TaskerMAN should support the re-allocation of a task & Create a new team member, first name: ``testy", last name ``mctesty", email: ``test@test.test" and a password of ``123". Create a task with: Title = ``make coffee", Start Date =``30/01/2016", End Date = ``1/02/2016", TaskElement1 title = ``get Cup", TaskElement2 title= ``grind beans". Allocate it to test@test.test. Go to edit tasks page, select the task and choose to re-allocate it to ``js@smith.com".  & The task should be updated in the remote database. & The show tasks page in TaskerMAN should now show the task with ``js@smith.com" allocated to it.  \\
\hline
MAN-014 & FR6 & TaskerMAN should support the abandonment of a task & Under the edit task page in TaskerMAN select to edit the task ``Make coffee", change its task status to ``Abandoned" & The task should be updated in the remote database. & The show tasks page in TaskerMAN should now show the task with a task status of ``Abandoned". \\
\hline
MAN-015 & FR7 & TaskerMAN should support the viewing of tasks. & Create a new task with: Title =  ``Hey Im a task look at me", Start Date = ``12/02/2016", End Date = ``12/02/2016", TaskElement1 = ``look at me pls" and allocate it to ``js@smith.com" (This is so that at least one task exists on the remote database). On TaskerMAN click on the ``View Tasks" page & A query should be sent to list all tasks in the database. & The task "Hey Im a task look at me" will be displayed, along with any other tasks should they exist in the remote database \\
\hline
MAN-016 & FR7 & TaskerMAN should support the filtering of tasks dependant on allocated team member. & On TaskerMAN in the ``view tasks" page select the option to filter by team member & A query should be sent to the remote database to list all the tasks by team member & The list of tasks that are displayed should be displayed by member in alphabetical ascending order. \\
\hline
MAN-017 & FR7 & TaskerMAN should support the filtering of tasks dependant on task status. & On TaskerMAN in the ``view tasks" page select the option to filter by task status & A query should be sent to the remote databse to list all the tasks by task status & The list of tasks that are displayed should be displayed by task status in alphabetical ascending order. \\
\hline
CLI-001 & FR8 & TaskerCLI should be able to identify the member upon logging in & Upon login, enter the user credentials: Username: ``jsmith@smith.com", Password: ``123". & A query should be sent to the database to locate the user with the data entered & The user will then be logged in and on the main page of TaskerCLI the current user will say ``logged in as ``Tom Smith"" \\
\hline
CLI-002 & FR8 & TaskerCLI should be able to identify the member upon logging in & Upon login enter in the credentials: Username = ``bob", Password = ``1" & Authenticator will check to see if the user in the database. & User should be notified that their credentials were incorrect. \\
\hline
CLI-003 & FR8 & TaskerCLI should be able to identify the member upon logging in & Upon login enter in the credentials: Email= ``DROP TABLE User" & A query will be run that partly contains the String ``DROP TABLE User". & The remote database should still be intact. \\
\hline
CLI-004 & FR8 & TaskerCLI should store the login details locally for future use. & Upon login enter in the credentials: Userame = ``js@smith.com" and the password = ``123", then logout. 
 & The users details will be stored in the local SQLite database (password should be hashed). & ``js@smith.com" should have their credentials stored in the local database (password also hashed).  \\
\hline
CLI-005 & FR8a & TaskerCLI should be able to store data on the user's computer locally for the user that is logged in. & Log in to the user ``js@smith.com" (password: 12345). & This should synchronise and store all data (locally) associated with ``js@smith.com". & The Local SQLite Database should contain the tasks for ``js@smith.com" \\
\hline
CLI-006 & FR9 & TaskerCLI should be able to synchronise to get all the tasks for the user that is logged in at the time and should only be ones that have a task status of ``allocated" & Create a new task on TaskerMAN with: Title = ``test code", Start date = ``30/01/2016", End date = ``01/02/2016", TaskElement1 = ``module testing", TaskElement2 = ``system testing". Allocate it to j``s@smith.com". On TaskerCLI log into the user ``js@smith.com" & This should synchronise with the remote database and display the list of tasks including the newly assigned task to ``js@smith.com".  & JTable displays added task. \\
\hline
CLI-007 & FR10 & Synchronisation should support the deletion of completed/abandoned tasks.  & Using TaskerMAN, select the task ``make coffee" and set its Task Status = ``ABANDONED". 
Login to ``js@smith.com" (on TaskerCLI)  & Upon logging in, TaskerCLI should synchronise with the remote database and no longer contain the ``make coffee" task. & The table will no longer contain the ``make coffee" task. \\
\hline
CLI-008 & FR9 & Synchronisation should support the deletion of completed/abandoned tasks. & Log into ``js@smith.com" on TaskerCLI, select the task ``test code" and select complete task. & After editing the task, TaskerCLI will synchronise with the remote database and should no longer display the ``test code" task.  & The completed task should be removed from the table.  \\
\hline
CLI-009 & FR10 & TaskerCLI must be able to support the local editing of tasks (Task element comments) & On TaskerMAN create a task with: Title = ``Do chores", Start Date = ``30/01/2016", End Date= ``1/02/2016", TaskElement title = ``Make a list of chores", TaskElement2 title= ``Find way of procrastinating". Allocate it to ``js@smith.com". Then login to TaskerCLI using ``js@smith.com". 
Edit the task ``Do chores" and enter the comment ``go for a pint" on the Task Element ``Find way of procrastinating". & TaskerCLI should synchronise with the database and update comment for the Task Element. & Upon selecting the task from now on the comment ``go for a pint" should remain associated with the element ``Find way of procrastinating".\\
\hline
CLI-010 & FR10 & TaskerCLI must be able to support the local editing of tasks (Task element comments) & Login to TaskerCLI with the user ``js@smith.com", edit the task ``Do chores" and enter the comment ``DROP TABLE Tasks" on the Task Element ``Make a list of chores".  & An SQL query will be executed which will partly contain the String ``DROP TABLE Tasks". & Remote database should still be intact. \\
\hline
CLI-011 & FR11 & Synchronisation must happen on startup & On TaskerMAN proceed to edit tasks, edit the task ``Make a list of chores" by changing the Task Element ``Find way of procrastinating" to ``do chores tomorrow instead". Login to TaskerCLI with ``js@smith.com" & TaskerCLI should synchronise with the database after logging in and update the Task Element. & After logging into TaskerCLI select the task ``Make a list of chores". The Task Element ``Find way of procrastinating" should now be ``do chores tomorrow instead". \\
\hline
CLI-012 & FR11 & Synchronisation must happen before editing a task. & Login to`` js@smith.com" on TaskerCLI. On TaskerMAN update the task ``Do chores" so that the Task Element ``Make a list of chores" is now ``Think about chores". Then select the task on TaskerCLI.  & After selecting the task, TaskerCLI should synchronise with the remote database and update the Task Element.  & After logging into TaskerCLI select the task ``Make a list of chores". The Task Element should now be called ``Think about chores". \\
\hline
CLI-013 & FR11 & Synchronisation must happen after editing a task. & Login to`` js@smith.com" on TaskerCLI. Select the task ``Do chores" and add the comment ``think harder" to the task element ``Think about chores", then close the task. & After selecting the task, TaskerCLI should synchronise with the remote database and update the Task Element. & The updated task element and comment should now be visible on TaskerMAN.  \\
\hline
CLI-014 & FR11 & TaskerCLI should synchronise with the database every 5 minutes of it being logged on. & Login to`` js@smith.com" on TaskerCLI. On TaskerMAN edit the task ``Think about chores" and set it's status to ``abandoned". Go back to TaskerCLI and wait for automatic sunchronisation to occur. & After 5 minutes TaskerCLI should synchronise.  & The task should no longer be visible on TaskerCLI after synchronisation. \\
\hline
UI-001 & IR1 & The user interfaces of TaskerMAN should be easy to use by regular computer users & A non-developer of the group try to create a user (for themselves) in TaskerMAN & User should be able to create a team member without asking for help & Team member has been created without any difficulty \\
\hline
UI-002 & IR1 & The user interfaces of TaskerMAN should be easy to use by regular computer users & 
A non-developer of the group try to edit a team member In TaskerMAN & User should be able to edit a team member without asking for help & The team member data can be edited by the user without any difficulty \\
\hline
UI-003 & IR1 & The user interfaces of TaskerMAN should be easy to use by regular computer users & A non-developer of the group try to delete a team (not themselves) member in TaskerMAN & User should be able to delete a team member without asking for help & The team member data should be deleted without any difficulty\\
\hline
UI-004 & IR1 & The user interfaces of TaskerMAN should be easy to use by regular computer users & A non-developer of the group try to create a new task in TaskerMAN and set the task to be allocated to them & User should be able to create a new task without asking for help & A task should be created by the user and allocated to themselves without any difficulty \\
\hline
UI-005 & IR1 & The user interfaces of TaskerMAN should be easy to use by regular computer users & 
A non-developer of the group try to edit the task that is allocated to them to have a task status of 'completed' in TaskerMAN & User should be able to edit a new task without asking for help & The user should be able to change the status of the task to "Completed" without any difficulty \\
\hline
UI-006 & IR1 & The user interfaces of TaskerMAN should be easy to use by regular computer users & 
A non-developer of the group try to delete a task (not the task that was allocated to them) TaskerMAN & 
User should be able to delete a new task without asking for help & The user should be able to delete the task without any difficulty \\
\hline
UI-007 & IR1 & The user interfaces of TaskerMAN should be easy to use by regular computer users & 
A non-developer of the group try to view members of the group & User should be able to view the team members in the group without asking for help & The user should be able to view the team members in the group without any difficulty \\
\hline
UI-008 & IR1 & The user interfaces of TaskerMAN should be easy to use by regular computer users & 
 non-developer of the group try to view the tasks & User should be able to view the tasks in the group, without asking for help & The user should be able to view the tasks allocated to group members without any difficulty \\
\hline
UI-009 & IR1 & The user interfaces of TaskerMAN should be easy to use by regular computer users & 
A non-developer of the group try to Filter tasks that are allocated to them & User should be able to see the task that was allocated to them, without asking for help & The user should be able to filter tasks that are allocated to them without any difficulty \\
\hline
UI-010 & IR1 & The user interfaces of TaskerMAN should be easy to use by regular computer users & A non-developer of the group try to Filter tasks that are completed & User should be able to see the task that they created (that has a task status of 'completed') & The user should be able to filter tasks that have been completed without any difficulty \\
\hline
UI-011 & IR1 & The user interfaces of TaskerCLI should be easy to use by regular computer users & A non-developer of the group try to login to TaskerCLI & User should be able to login without asking for help & The user should be able to login to TaskerCLI without any issue \\
\hline
UI-012 & IR1 & The user interfaces of TaskerCLI should be easy to use by regular computer users & A non-developer of the group try to view tasks in TaskerCLI & User should be able to navigate to be able to view tasks without asking for help & The user should be able to view the tasks and their respective task elements without any difficulty \\
\hline
UI-013 & IR1 & The user interfaces of TaskerCLI should be easy to use by regular computer users & A non-developer of the group try to edit a task in TaskerCLI & User should be able to edit a task without asking for help & The user should be able to edit a task without any difficulty \\
\hline
UI-014 & IR1 & The user interfaces of TaskerCLI should be easy to use by regular computer users & A non-developer of the group try to use the search bar to search for a task & User should be able to search for a task using the search bar without asking for help & The user should be able to use the search bar to search for a specific task without any difficulty \\
\hline
UI-015 & IR1 & The user interfaces of TaskerCLI should be easy to use by regular computer users & A non-developer of the group try to filter tasks by those allocated to them & User should be able to filter tasks that were allocated to them without asking for help & The user should be able to filter the tasks without any difficulty \\
\hline
PER-001 & PR1 & Program should respond to user input in a minimum of a second & Enter a user's credentials into the login page of TaskerCLI and click enter & The page should log in in less than a second & The page logged in in less than a second \\
\hline
PER-002 & PR1 & Program should respond to the user input in a minimum of a second & Press show tasks in TaskerCLI & The tasks should be displayed in less than a second & The tasks were displayed in less than a second \\
\hline
PER-003 & PR1 & Program should respond to the user input in a minimum of a second & Enter data for a new team member in TaskerMAN & The member should be visible within one second & The user was visible within one second \\
\hline
PER-004 & PR1 & Program should respond to the user input in a minimum of a second & Enter data for a new Task in TaskerMAN & The task should be visible within one second & The task was visible within one second \\
\hline
PER-005 & PR2 & TaskerCLI should run on any machine supporting Java & Run TaskerCLI on a Linux machine & Program should execute & Program executed \\
\hline
PER-006 & PR2 & TaskerCLI should run on any machine supporting Java & Run TaskerCLI on a Windows machine & Program should execute & Program executed \\
\hline
PER-007 & PR2 & TaskerCLI should run on any machine supporting Java & Run TaskerCLI on a Mac & Program should execute & Program executed \\
\hline
PER-008 & PR2 & TaskerSRV must run on a suitable web server & Test connection to database using JDBC by logging into TaskerCLI & Should connect successfully & JDBC connected to the database successfully \\
\hline
PER-009 & PR2 & TaskerMAN should be deployable on an apache web server & Deploy TaskerMAN to an apache web server & Should work correctly & Worked correctly \\
\hline
\end{longtable}
\clearpage
\addcontentsline{toc}{section}{REFERENCES}
\begin{thebibliography}{2}
\bibitem{se.qa.rs} \emph{Software Engineering Group Projects}
Requirements Specifications.
N. W. Hardy, SE.QA.RS. 1.2 Release.
\bibitem{se.qa.tps} \emph{Software Engineering Group Projects}
Test Procedure Standards
C. J. Price, N.W.Hardy and B.P.Tiddeman, SE.QA.06. 1.8 Release.
\end{thebibliography}
\addcontentsline{toc}{section}{DOCUMENT HISTORY}
\section*{DOCUMENT HISTORY}
\begin{tabular}{|l | l | l | p{8cm} |l | }
\hline
Version & CCF No. & Date & Changes made to Document & Changed by \\
\hline
1.0 & N/A & 2015-11-12 & Initial creation for LaTeX & L. Jones \\
\hline
1.1 & N/A & 2016-14-02 & Updated by Tom M based on Nigel's feedback, final version & L. Jones \\
\hline
\end{tabular}
\label{thelastpage}
\end{document}
\end{verbatim}
\label{fig:footer}
\end{figure}
\label{thelastpage}
\end{document}
