\documentclass{project}
\usepackage[pdfauthor={L Jones},pdftitle={Software Engineering Group Project, Interaction and high level design for the system},pdftex]{hyperref}
\usepackage{graphicx}
\usepackage[export]{adjustbox}
\usepackage{longtable}
\graphicspath{ {images} }
\begin{document}
\title{Software Engineering Group Project}
\subtitle{Test Specification}
\author{L. Jones, T. Oram, W. Jones, T. MIlls}     
\shorttitle{Test Specification}
\version{1.0}
\status{Release}
\date{2015-11-12}
\configref{SE-12-TS}
\maketitle
\tableofcontents
\newpage
\section{INTRODUCTION}
\subsection{Purpose of this document}
The purpose of this document is to outline how we aim to enforce good testing practice, and as such contains a 'Test Specification' which demonstrates how to test against the different requirements as specified in the requirements specification\cite{se.qa.rs}, as well as displaying the criteria that must be met in order to pass tests.
\subsection{Scope}
This document specifically covers the 'Test Specification' section of the Test Procedure Standards Quality Assurance Document\cite{se.qa.tps}\\
\newline
This document should be read by all project members. It is assumed that the reader is already familiar with the Test Procedure Standards \cite{se.qa.tps}.
\subsection{Objectives}
The objectives of this document are as follows:
\begin{itemize}
	\item Identify where a test will be necessary against the requirements specification{se.qa.rs} and give it a unique ID. 
	\item Explain the exact function/requirement that is being tested.
	\item Explain how a user would undertake a test, as well as the desired output if successful.
	\item Explain the criteria that must be met by the test in order for it to be deemed as "Passed".
\end{itemize} 
\clearpage
\section{TEST SPECIFICATION}
\begin{longtable}{| p{2cm} | p{1.5cm} | p{2.8cm} | p{2.8cm} | p{2.8cm} | p{2.8cm} |}
\hline
Test Reference & Reference Requirement & Test Content & Input & Output & Pass Criteria \\
\hline
SRV-001 & FR1 & To check that the system can create two pieces of information; full name and email address & In an SQL command line insert into the team member table & The table should now contain the entered data & The entered data should be returned when a query is run.  \\
\hline
SRV-002 & FR1 & To check that the database can support the updating of team member data & In an SQL command line; update the data in the team member table & The table should now contain the updated data & The data will be updated to the new data that was entered\\
\hline
SRV-003 & FR1 & To check that the system supports deletion of team members & In an SQL command line; delete the data in the team member table & The table should not include the member that was deleted & The table should not have a record of that team member\\
\hline
SRV-004 & FR2 & The system must be able to support the creation of a task & In an SQL Command Line; Create a task setting the title, whom it is allocated to, start and end dates, the task elements and the task status and then go to the "display tasks" page & The task table should now have one more record & The task will have been created and will consist of all the data that was entered in before\\
\hline
SRV-005 & FR2 & The system must be able to support the updating of a task & In an SQL command line; update the task data to have a task status of 'allocated' and an end date of 21/10/2016 & The data should be updated & The task should have the new end date and status \\
\hline
SRV-006 & FR2 & The system must be able to support the deletion of tasks & In an SQL command line; Delete a record in the task table & The task table should not include the record that was removed from the database & The task will be deleted from the database\\
\hline
MAN-001 & FR3 & TaskerMAN should support the creating of team member data & In the Create user page of taskerMAN create a new user with name "John Smith" and email "js@aber.ac.uk"
 & The Show users page in TaskerMAN will now include the user "John Smith" & All data entered should be correct and the show users page displays the member \\
\hline
MAN-002 & FR3 & TaskerMAN should support the updating of a team member & UUnder the edit user page of TaskerMAN edit users name to "Tom Smith" & The show users page in TaskerMAN should now have the updated task member & The task member will now have a name of "Tom Smith" \\
\hline
MAN-003 & FR3 & TaskerMAN should support the deletion of a team member & Under the edit user page of TaskerMAN delete the user with the name "Tom Smith" & The user will be deleted from the database & The show users page in TaskerMAN should no longer include the team member that was deleted \\
\hline
MAN-004 & FR4 & TaskerMAN should support the creation of task data & Under the create task page of TaskerMAN create a task where the task title is "Make coffee" with a start date of ?20/10/2015? and an end date of '21/10/2015' also set the member assigned and also the tasks status & The task should be added to the database & The show tasks page in taskerMAN will show the task entered and should have the exact data entered \\
\hline
MAN-005 & FR4 & TaskerMAN should support the updating of task data & Under the edit task page in TaskerMAN select any task to update and update the task status to 'completed' & The database should be updated & The show tasks page in TaskerMAN will show the updated task and its task status should now be 'completed' \\
\hline
MAN-006 & FR4 & TaskerMAN should support the deleting of task data & Under the edit task page in TaskerMAN, select any task and then delete it & The selected task will be deleted from the database & The show tasks page in TaskerMAN should no longer display the task that was deleted \\
\hline
MAN-007 & FR5 & TaskerMAN should support the re-allocating of a task & Under the edit task page in TaskerMAN select any task to edit. Next; change the member allocated to "John Smith"  & The task should be updated in the database & The show tasks page in TaskerMAN should now show the task with "John Smith" allocated to it \\
\hline
MAN-008 & FR6 & TaskerMAN should support the abandoning of a task & Under the edit task page in TaskerMAN select any task to edit. Next; change the task status to "Abandoned" & The task should be updated in the database & The show tasks page in TaskerMAN should now show the task with a task status of 'Abandoned' \\
\hline
MAN-009 & FR7 & TaskerMAN should support the viewing of tasks & On TaskerMAN click "View Tasks" & A query should be sent to list all tasks in the database & A list of every task will be displayed \\
\hline
MAN-010 & FR7 & TaskerMAN should support the filtering of tasks dependant on allocated team member & On TaskerMAN in the "view tasks" page select the option to filter by team member and then type "John Smith"  & A query should be sent to list all the tasks that are allocated to "John Smith" & The list of tasks that are displayed should only be allocated to "John Smith" \\
\hline
MAN-011 & FR7 & TaskerMAN should support the filtering of tasks dependant on the task status & On TaskerMAN in the "view tasks" page select the option to filter by task status and select to show all the tasks that are 'completed' & A query should be sent to list all the tasks that have a task status of 'completed' & The list of tasks that are displayed should only have a task status of 'completed' \\
\hline
CLI-001 & FR8 & TaskerCLI should be able to identify the member upon logging in & Upon login, type an admins username and password and then press the login button Username: jsmith
Password: password; & A query should be sent to the database to locate the user with the data entered & The user will then be logged in and on the main page of TaskerCLI the current user will say "logged in as "jsmith"" \\
\hline
CLI-002 & FR8 & TaskerCLI should store the login details locally for future use & Upon login select the "remember me" checkbox and enter the details: 
UserName: 
Jsmith
Password: password & Will run a query that will return a record with matching credentials and will be stored locally & When the program is restarted the user will be automatically logged in \\
\hline
CLI-003 & FR8a & TaskerCLI should be able to store data on the users computer locally for the user that is logged in & Login to a user (this will synchronise) & After synchronisation the program should save all the tasks that are allocated to them locally to a text file (located in the users home directory) & The text file will now display a table showing the tasks that were allocated to the member logged in
 \\
\hline
CLI-004 & FR9 & TaskerCLI should be able to synchronise to get all the tasks for the user that is logged in at the time and should only be ones that have a task status of 'allocated' & Login to a user that has tasks allocated to them with a status of 'allocated'. Then press the synchronisation button & A query will be ran to return all of the tasks that are allocated to the user and those tasks will be stored in a text file locally in the users home directory & The text file will contain a list of tasks that will be identical to those stored in the database that are allocated to the user that is logged in at the time and has a task status of 'allocated' \\
\hline
CLI-005 & FR9 & Synchronisation should support the deletion of completed/abandoned tasks & Login to a user that already has a text file containing tasks allocated to them with a task status of 'allocated' and then edit one of the tasks so that its task status is now 'completed'. Also edit one of the tasks so that its status is 'abandoned' & A query will be ran to return all of the tasks that are allocated to the user and those tasks will overwrite the existing text file (removing any completed/abandoned tasks) & The text file will not display the task that was set to ?completed? and it will also not display the task that was set to 'abandoned' \\
\hline
CLI-006 & FR10 & TaskerCLI must be able to support local editing of tasks. 
(Task status) & Select a task from the list of tasks and then select to edit that task. Change the tasks status to 'completed' then proceed to view tasks & The task will have been updated in the database & In the show tasks panel the task that was edited will now have a task status of 'completed' \\
\hline
CLI-007 & FR10 & TaskerCLI must be able to support the local editing of tasks. (task element note) & Select a task from the list of tasks and then select to edit that task. Edit one of the task elements notes to something different & The task element should have been updated in the database & The show tasks panel will now display the task with the updated task note \\
\hline
CLI-008 & FR11 & Synchronisation must happen on start up & After logging in TaskerCLI will start and synchronisation will happen at this time. Check the local files to see if it has displayed all the current allocated tasks for that member in a text file & The TaskerCLI should synchronise with the database and the local files will be updated & The local files should be updated straight after login \\
\hline
CLI-009 & FR11 & Synchronisation must happen before editing a task & Edit a task in the database with SQL by setting its task status to 'completed'. Next; select the task (this should not show the updated task status yet) and click edit task & TaskerCLI should synchronise with the database and display the task with the updated task status that were input using SQL & Within the field boxes on the form in the edit task panel of TaskerCLI the newly updated task will be displayed showing the task status as 'completed' \\
\hline
CLI-010 & FR11 & Synchronisation must happen after editing a task & Login to a user that has tasks allocated to them and select to edit one of the tasks and change its task status to 'completed' Press update. Proceed to check the database for this task to see if it has been updated by running a query & TaskerCLI should synchronise with the database and the new task will be shown in the database after the SQL query is ran & TaskerCLI will have synchronised and the query will return the field with the updated task status \\
\hline
CLI-011 & FR11 & TaskerCLI should synchronise with the database every 5 minutes of it being logged on & Login to TaskerCLI and record the time of start. Update a task in the database whilst waiting for the 5 minutes to pass. Next: Refresh the show tasks panel after 5 minutes have passed from the start time & After 5 minutes TaskerCLI should synchronise with the database and store the new updated task in the show tasks panel & The updated task will be displayed in the show tasks panel after refreshing when 5 minutes have passed \\
\hline
UI-001 & IR1 & The user interfaces of TaskerMAN should be easy to use by regular computer users & A non-developer of the group try to create a user (for themselves) in TaskerMAN & User should be able to create a team member without asking for help & It was obvious to the user \\
\hline
UI-002 & IR1 & The user interfaces of TaskerMAN should be easy to use by regular computer users & 
A non-developer of the group try to edit a team member In TaskerMAN & User should be able to edit a team member without asking for help & It was obvious to the user \\
\hline
UI-003 & IR1 & The user interfaces of TaskerMAN should be easy to use by regular computer users & A non-developer of the group try to delete a team (not themselves) member in TaskerMAN & User should be able to delete a team member without asking for help & It was obvious to the user \\
\hline
UI-004 & IR1 & The user interfaces of TaskerMAN should be easy to use by regular computer users & A non-developer of the group try to create a new task in TaskerMAN and set the task to be allocated to them & User should be able to create a new task without asking for help & It was obvious to the user \\
\hline
UI-005 & IR1 & The user interfaces of TaskerMAN should be easy to use by regular computer users & 
A non-developer of the group try to edit the task that is allocated to them to have a task status of 'completed' in TaskerMAN & User should be able to edit a new task without asking for help & It was obvious to the user \\
\hline
UI-006 & IR1 & The user interfaces of TaskerMAN should be easy to use by regular computer users & 
A non-developer of the group try to delete a task (not the task that was allocated to them) TaskerMAN & 
User should be able to delete a new task without asking for help & It was obvious to the user \\
\hline
UI-007 & IR1 & The user interfaces of TaskerMAN should be easy to use by regular computer users & 
A non-developer of the group try to view members of the group & User should be able to view the team members in the group without asking for help & It was obvious to the user \\
\hline
UI-008 & IR1 & The user interfaces of TaskerMAN should be easy to use by regular computer users & 
 non-developer of the group try to view the tasks & User should be able to view the tasks in the group, without asking for help & It was obvious to the user \\
\hline
UI-009 & IR1 & The user interfaces of TaskerMAN should be easy to use by regular computer users & 
A non-developer of the group try to Filter tasks that are allocated to them & User should be able to see the task that was allocated to them, without asking for help & It was obvious to the user \\
\hline
UI-010 & IR1 & The user interfaces of TaskerMAN should be easy to use by regular computer users & A non-developer of the group try to Filter tasks that are completed & User should be able to see the task that they created (that has a task status of 'completed') & It was obvious to the user \\
\hline
UI-011 & IR1 & The user interfaces of TaskerCLI should be easy to use by regular computer users & A non-developer of the group try to login to TaskerCLI & User should be able to login without asking for help & It was obvious to the user \\
\hline
UI-012 & IR1 & The user interfaces of TaskerCLI should be easy to use by regular computer users & A non-developer of the group try to view tasks in TaskerCLI & User should be able to navigate to be able to view tasks without asking for help & It was obvious to the user \\
\hline
UI-013 & IR1 & The user interfaces of TaskerCLI should be easy to use by regular computer users & A non-developer of the group try to edit a task in TaskerCLI & User should be able to edit a task without asking for help & It was obvious to the user \\
\hline
UI-014 & IR1 & The user interfaces of TaskerCLI should be easy to use by regular computer users & A non-developer of the group try to use the search bar to search for a task & User should be able to search for a task using the search bar without asking for help & It was obvious to the user \\
\hline
UI-015 & IR1 & The user interfaces of TaskerCLI should be easy to use by regular computer users & A non-developer of the group try to filter tasks by those allocated to them & User should be able to filter tasks that were allocated to them without asking for help & It was obvious to the user \\
\hline
UI-016 & IR1 & The user interfaces of TaskerCLI should be easy to use by regular computer users & A non-developer of the group try to synchronise & User should be able to synchronise TaskerCLI with the database without asking for help & It was obvious to the user \\
\hline
PER-001 & PR1 & Program should respond to user input in a minimum of a second & Enter a user's credentials into the login page of TaskerCLI and click enter & The page should log in in less than a second & The page logged in in less than a second \\
\hline
PER-002 & PR1 & Program should respond to the user input in a minimum of a second & Press show tasks in TaskerCLI & The tasks should be displayed in less than a second & The tasks were displayed in less than a second \\
\hline
PER-003 & PR1 & Program should respond to the user input in a minimum of a second & Enter data for a new team member in TaskerMAN & The member should be visible within one second & The user was visible within one second \\
\hline
PER-004 & PR1 & Program should respond to the user input in a minimum of a second & Enter data for a new Task in TaskerMAN & The task should be visible within one second & The task was visible within one second \\
\hline
PER-005 & PR2 & TaskerCLI should run on any machine supporting Java & Run TaskerCLI on a Linux machine & Program should execute & Program executed \\
\hline
PER-006 & PR2 & TaskerCLI should run on any machine supporting Java & Run TaskerCLI on a Windows machine & Program should execute & Program executed \\
\hline
PER-007 & PR2 & TaskerCLI should run on any machine supporting Java & Run TaskerCLI on a Mac & Program should execute & Program executed \\
\hline
PER-008 & PR2 & TaskerSRV must run on a suitable web server & Test connection to database using JDBC by logging into TaskerCLI & Should connect successfully & JDBC connected to the database successfully \\
\hline
PER-009 & PR2 & TaskerMAN should be deployable on an apache web server & Deploy TaskerMAN to an apache web server & Should work correctly & Worked correctly \\
\hline
DES-001 & DC1 & TaskerCLI must be written in Java & Check over code make sure it is written in the correct standard of Java & Should be written in the correct standard & Code was valid\\
\hline
DES-002 & DC2 & TaskerMAN must be written in any Server side language & Check over code to make sure it is written in the correct standard of PHP and JS & Should be written in the correct standard & Code was valid \\
\hline
DES-003 & DC2 & The system should be able to work with a constraint of 20 members with 30 allocated tasks & Enter into the database 20 members and 30 allocated tasks. Use basic functionality of the program and check for lag & There should be little lag (still working within one second from user input) & No/little latency \\
\hline
\end{longtable}
\clearpage
\addcontentsline{toc}{section}{REFERENCES}
\begin{thebibliography}{2}
\bibitem{se.qa.rs} \emph{Software Engineering Group Projects}
Requirements Specifications.
N. W. Hardy, SE.QA.RS. 1.2 Release.
\bibitem{se.qa.tps} \emph{Software Engineering Group Projects}
Test Procedure Standards
C. J. Price, N.W.Hardy and B.P.Tiddeman, SE.QA.06. 1.8 Release.
\end{thebibliography}
\addcontentsline{toc}{section}{DOCUMENT HISTORY}
\section*{DOCUMENT HISTORY}
\begin{tabular}{|l | l | l | p{8cm} |l | }
\hline
Version & CCF No. & Date & Changes made to Document & Changed by \\
\hline
1.0 & N/A & 2015-11-12 & Initial creation for LaTeX & L. Jones \\
\hline
\end{tabular}
\label{thelastpage}
\end{document}
\end{verbatim}
\label{fig:footer}
\end{figure}
\label{thelastpage}
\end{document}